\documentclass{article}
\begin{document}
\title{Homework 2}
\author{Riccardo Cereghino - S4651066}
\date{25-3-2020}
\maketitle

\paragraph{2.1}%
\label{par:2.1}
$(i)$ La proprietà segue dal fatto che $E^CE=\emptyset$ e $E^C\cup E=S$, quindi
$P(E)+P(E^C)=1$.

$(ii)$ La proprietà si dimostra notando che $P(EF)=P(E)$ che si può scrivere
come $F=E\cup E^CF$, da cui possiamo derivare $P(F)=P(E)+\frac{P(E^C F}{\geq0}$.

 $(iii)$
\[P(A\cup B)=P(A)+P(B)-P(A\cap B)\]
In primo luogo,

\[P(A\cup B)=P(A)+P(B\setminus A)\]
 (per il terzo Assioma), quindi,
\[P(A\cup B)=P(A)+P(B\setminus (A\cap B))\]
perché
 \[B\setminus A=B\setminus (A\cap B)\]
E,
\[P(B)=P(B\setminus (A\cap B))+P(A\cap B)\]
sottraendo 
\[P(B\setminus (A\cap B))\]
da entrambe le equazioni otteniamo il risultato voluto.

\paragraph{2.2}%
\label{par:2.2}
\[C(3,10)= \frac{10!}{3!7!}=120 \]
\[C(2,5)= \frac{5!}{2!3!}=10 \]
\[C_{TOT}=120*10=1200\]

\paragraph{2.3}%
\label{par:2.3}
\subparagraph{Poker}%
\label{subp:Poker}
\textbf{Casi possibili:} $C(5,52)^ 2598960$

\textbf{Casi favorevoli:} $13\times48=624$ dove $13$ dei tipi
possibili e $48$ sono le possibili quinte carte rimanenti.

\[P= \frac{624}{2598960}=0.024\%\]

\subparagraph{Scala reale}%
\label{subp:Scala reale}
\textbf{Casi possibili:} $C(5,52)^ 2598960$

\textbf{Casi favorevoli:} $4\times10=40$ dove $4$ è il numero di semi e 
$10$ il numero di progressioni possibili.

\[P= \frac{40}{2598960}=0.0015\%\]

\paragraph{2.4}%
\label{par:2.4}
Calcolo la probabilità di ottenere $3$ palline pari su $30$
\textbf{Casi possibili:} $C(30,3)= \frac{30!}{3!27!} = 4060$
\textbf{Casi favorevoli:} $C(15,3) \frac{15!}{3!12!} = 455$
$P_1=13/116$

Calcolo la probabilità di ottenere $4$ palline minori di $10$
\textbf{Casi possibili:} $C(27,4)= 17550$
\textbf{Casi favorevoli:} $C(9,4) = 126$
$P_2=7/975$

Quindi calcolo la probabilità totale considerando che dovrò togliere i casi
ripetuti, ovvero $4/30$
\[P(E)=P_1+P_2-2/15=xxx\]


\paragraph{2.5}%
\label{par:2.5}
\textbf{Casi possibilie}: $C(10,2)= 45$
\textbf{Casi favorevoli:} $C(6,1) C(4,1)= 24$

$P_{TOT}=24/45$
\end{document}
