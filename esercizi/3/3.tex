\documentclass{article}
\begin{document}
\title{Homework 3}
\author{Riccardo Cereghino - S4651066}
\maketitle

\paragraph{3.1}%
\label{par:3.1}
Siano $V$ gli studenti che hanno passato l'esame, $X$ quelli che non lo hanno
passato, $P$ gli studenti preparati e $I$ gli impreparati.
\subparagraph{a}%
\label{subp:a}
\[
  P(V)=P(P)P(V|P)+P(I)P(V|I)= \frac{70}{100}\frac{90}{100}+\frac{30}{100}
  \frac{2}{100}=64\%   
\]
\subparagraph{b}%
\label{subp:b}
\[
  P(V\cap I)=P(I)P(V|I)= \frac{30}{100} \frac{2}{100}=0.06\%
\]
\[
  P(I|V)= \frac{P(V\cap I)}{P(I)}= \frac{0.6\%}{64\%}=0.94\%
\]
\paragraph{3.2}%
\label{par:3.2}
Sia $T1$ cliente di tipo 1, $T2$ di tipo 2 e $T3$ di tipo 3, $H$ se il bersaglio
è stato colpito e $M$ se mancato.
\subparagraph{a}%
\label{subp:a}
\[
  P(V)=P(T1)P(H1|T1)+P(T2)P(H2|T2)+P(T3)P(H3|T3)=
  \frac{8}{100}+\frac{15}{100}+\frac{12}{100}=35\%
\]

\paragraph{3.3}%
\label{par:3.3}
Sia $I$ il $2\%$ della popolazione infettata, $T$ il fatto che il test rilevi la
patologia il $98\%$ delle volte e $E$ la parcentuale di errore del test del
$1\%$.

\[
  P(E)=P(I)*P(T|I)*P(\overline{E}|T|I)=1.94\%
\]

\paragraph{3.4}%
\label{par:3.4}
Sia $H$ testa.
\[
  P(H)=P(A)P(H|A)+P(B)P(H|B)= \frac{8}{15} 
\]
\[
  P(A|H)= \frac{P(A\cap H)}{P(H)}= \frac{1}{2} 
\]

\paragraph{3.5}%
\label{par:3.5}
\textbf{Ipotesi:} se $A$ e $B$ sono indipendenti non sono mutualmente esclusivi.

\textbf{Dimostrazione:} dato che $A$ e $B$ sono indipendenti la loro probabilità
congiunta si può scrivere come $P(A\cap B)=P(A)P(B)$, ma per ipotesi questi sono
a probabilità non nulola, per cui diversi da $0$, quindi non possono essere
mutualmente esclusivi in quanto lo saranno solo nel caso $A\cap B=\emptyset$.

\paragraph{3.6}%
\label{par:3.6}
Sia $O$ il dado onesto.
\[
  P(1)=P(2)=P(O)P(1|O)+P(\overline{O})+P(1|\overline{O}= \frac{11}{90}
\]
\[
  P(1\cup2)=P(1)+P(2)= \frac{22}{90} 
\]
\[
  P(6)=P(O)P(6|O)+P(\overline{O})+P(6|\overline{O}= \frac{19}{90}=21\%
\]
\[
  P(2\cup6)=P(2)+P(6)= \frac{1}{3}=33\% 
\]

\end{document}
