\documentclass{article}
\usepackage[T1]{fontenc}
\usepackage[utf8]{inputenc}
\usepackage{textcomp}
\usepackage{eurosym}

\usepackage{import}
\usepackage{pdfpages}
\usepackage{transparent}
\usepackage{xcolor}
\usepackage{wrapfig}
\usepackage{amsmath}

\newcommand{\incfig}[2][1]{%
    \def\svgwidth{#1\columnwidth}
    \import{./figures/}{#2.pdf_tex}
}

\pdfsuppresswarningpagegroup=1

\inputencoding{utf8}

\begin{document}
\title{Homework 3}
\author{Riccardo Cereghino - S4651066}
\maketitle

\paragraph{4.2}%
\label{par:4.2}
Siano $V$ la pallina verde, $B$ la bianca e $R$ la rossa, ed $X$ la vincita (o
perdita) basata sulla pescata, ovvero la variabile casuale.

\subparagraph{a}%
\label{subp:a}
Analizziamo la distribuzione delle vincite e delle perdite in base alle palline
pescate:
\begin{itemize}
  \item per $X=-3$ abbiamo un solo caso, $RRR$;
  \item $X=-2$ abbiamo il caso $BRR$;
  \item $X=-1$ abbiamo i casi $BBR$ e $VRR$;
  \item $X=0$ abbiamo i casi $BBB$ e $VBR$;
  \item $X=1$ abbiamo i casi $VBB$ e $VVR$;
  \item $X=2$ il caso $VVB$.
\end{itemize}

Possiamo quindi disegnare la funzione di probabilità di massa.
\begin{figure}[ht]
    \centering
    \incfig[0.6]{funzione-massa-1}
    \caption{Funzione di massa}
    \label{fig:funzione-massa-1}
\end{figure}

Mentre il valore atteso della vincita sarà dato dalla seguente formula, dove
$P(i)$ sarà dato dai casi favorevoli per $X=i$ ed i casi possibili ($9$).
\[
  \begin{split}
  E[x]&= \sum^{2}_{i=-3} a_i P(i)=\\
  &= \frac{-3}{9}-\frac{2}{9}-\frac{2}{9}+\frac{2}{9}+\frac{2}{9}=\\
  &= \frac{-2}{9} = -0.22\text{\euro}
  \end{split}
\]

\subparagraph{b}%
\label{subp:b}
Analizziamo la distribuzione delle vincite e delle perdite in base alle palline
pescate:
\begin{itemize}
  \item $X=-2$ abbiamo il caso $BRR$;
  \item $X=-1$ abbiamo i casi $BBR$ e $VRR$;
  \item $X=0$ abbiamo i casi $BBB$ e $VBR$;
  \item $X=1$ abbiamo i casi $VBB$ e $VVR$;
  \item $X=2$ il caso $VVB$.
\end{itemize}

Possiamo quindi disegnare la funzione di probabilità di massa.
\begin{figure}[ht]
    \centering
    \incfig[0.6]{funzione-massa-2}
    \caption{Funzione di massa}
    \label{fig:funzione-massa-2}
\end{figure}

Mentre il valore atteso della vincita sarà dato dalla seguente formula, dove
$P(i)$ sarà dato dai casi favorevoli per $X=i$ ed i casi possibili ($8$).
\[
  \begin{split}
  E[x]&= \sum^{2}_{i=-3} a_i P(i)=\\
  &= -\frac{2}{8}-\frac{2}{8}+\frac{2}{8}+\frac{2}{8}=0\text{\euro}
  \end{split}
\]
\end{document}
