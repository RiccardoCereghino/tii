\section{Definizione assiomatica di probabilità}%
\label{sec:Definizione assiomatica di probabilità}
Definiamo la probabilità in modo assiomatico discutento alcune proprietà nel
caso semplice, ma importante in cui sia possibile individuare eventi
equiprobabili.

\subsection{Nozioni fondamentali}%
\label{sub:Nozioni fondamentali}
\begin{itemize}
  \item \textbf{Spazio campionario:} è l'insieme $S$ dei possibili risultati di
    un esperimento;
  \item \textbf{evento:} è un qualunque sottoinsieme $E$ di $S$ che si realizza
    se il risultato dell'esperimento appartiene ad $E$.
\end{itemize}

Indichiamo gli eventi di:
\begin{itemize}
  \item \textbf{unione:} come $E\cup F$;
  \item \textbf{intersezione:} $EF$;
  \item \textbf{mutualmente esclusivi:} $EF=\emptyset$;
  \item \textbf{complementare:} $E^C$ tale che $E\cup E^C=S$.
\end{itemize}

\subsection{Assiomi}%
\label{sub:Assiomi}
Una probabilità $P(.)$ risulta ben definita sugli eventi di uno spazio
campionario $S$ se:
\begin{equation}
  \begin{split}
    A1&:0\leq P(E)\leq1\qquad\forall E\subseteq S\\
    A2&:P(S)=1\\
    A3&: \text{se }E_i,i=1,2,\dots \text{mutualmente esclusivi}\rightarrow
    P(\cup E_i)=\sum_i P(E_i)
  \end{split}
\end{equation}

Sono conseguenze di $A1$, $A2$, $A3$:
\begin{equation}
  \begin{split}
    (i)&:\forall E,P(E^C)=1-P(E)\\
    (ii)&:E\subseteq F\rightarrow P(E)\leq P(F)\qquad\forall E,F\\
    (iii)&: P(E\cup F)=P(E)+P(F)-P(EF)\qquad\forall E,F
  \end{split}
\end{equation}

\paragraph{Dimostrazione}%
\label{par:Dimostrazione}
$(i)$ La proprietà segue dal fatto che $E^CE=\emptyset$ e $E^C\cup E=S$, quindi
$P(E)+P(E^C)=1$.

$(ii)$ La proprietà si dimostra notando che $P(EF)=P(E)$ che si può scrivere
come $F=E\cup E^CF$, da cui possiamo derivare $P(F)=P(E)+\frac{P(E^C F}{\geq0}$.

 $(iii)$
\[P(A\cup B)=P(A)+P(B)-P(A\cap B)\]
In primo luogo,

\[P(A\cup B)=P(A)+P(B\setminus A)\]
 (per il terzo Assioma), quindi,
\[P(A\cup B)=P(A)+P(B\setminus (A\cap B))\]
perché
 \[B\setminus A=B\setminus (A\cap B)\]
E,
\[P(B)=P(B\setminus (A\cap B))+P(A\cap B)\]
sottraendo 
\[P(B\setminus (A\cap B))\]
da entrambe le equazioni otteniamo il risultato voluto.

\subsection{Eventi equiprobabili}%
\label{sub:Eventi equiprobabili}
Sia $S$ lo \emph{spazio campionario} costituito da un insieme finito di $N$
risultati che indichiamo con i primi $\N$ numeri naturali (\emph{cardinalità}),
ovvero $S=\{1,2,\dots,N\}$, sia $\#S$ la cardinalità dell'insieme.
Se le probabilità $P(i)$ sono tutte uguali allora $P(\{i\})=\dfrac{1}{N}$.

La probabilità di un insieme $E\subseteq S$ in questo caso sarà:
\[
  P(E)=\frac{\#E}{\#S}
\]

\begin{esercizio}[Ottenere 7 lanciando due dadi]

  \text{Casi possibili}: $6\times6$.
  \text{Casi favorevoli}: $\{6,1\},\{1,6\},\{5,2\},\{2,5\},\{4,3\},\{3,4\}$.
  \[
    P(E)=\frac{\#E}{\#S}=\frac{6}{36}= \frac{1}{6} 
  \]
\end{esercizio}

\begin{esercizio}[Due persone nate nello stesso giorno]
  \text{Casi possibili}: $\#E=365!$.
  \text{Casi favorevoli}: $\#S=365^{n-1}$. 
  \[
    P(E)=\frac{\#E}{\#S}=\frac{365!}{365^{n-1}}
  \]
\end{esercizio}

\chapter{Probabilità condizionata e indipendenza}%
\label{sec:Probabilità condizionata e indipendenza}
Dati due eventi $E$ ed $F$ siamo interessati a calcolare la probabilità di $E$
quando sappiamo che si è realizzato $F$.

Sia $P(E|F)$ la probabilità di $E$ condizionata a $F$.
\[
  P(E|F)= \frac{P(EF)}{P(F)} 
\]

Si osservi che:
\begin{itemize}
  \item $0<P(E|F)<1$
  \item $P(F|F)=1$
  \item $E_i,E_j=0;P(\cup E_i|F)=\sum P(E_i|F)$
\end{itemize}

\paragraph{Regola di moltiplicazione}%
\label{par:Regola di moltiplicazione}
Iniziamo scrivendo:
\[
  P(E|F)= \frac{P(EF)}{P(F)}\Rightarrow P(EF)=P(E|F)P(F)
\]

Dati $E_1,E_2,E_3$
\[
  P(E_1 E_2 E_3)=P(E_1|E_2 E_3)P(E_2 E_3)
\]
\[
  P(E_1 E_2 E_3)=P(E_1|E_2 E_3)P(E_2|E_3)P(E_3)
\]

\section{Teorema di Bayes}%
\label{sec:Teorema di Bayes}
\[
  P(E|F)= \frac{P(F|E)P(E)}{P(F)} 
\]

Formula della probabilità assoluta:
\[
  P(F)=P(F)P(E|F)+P(F^C)P(F|E^C)
\]

Da cui:
\[
  P(E|F)= \frac{P(F|E)P(E)}{P(F)P(E|F)+P(F^C)P(F|E^C)
} 
\]

\subsection{Eventi indipendenti}%
\label{sub:Eventi indipendenti}

\section{Variabili casuali discrete}%
\label{sec:Variabili casuali discrete}
Sia $S$ lo spazio campionario, $X:\rightarrow\R$
\[
  (S,P)\xrightarrow[X](\R,\_)
\]

\begin{definizione}[Variabile causale (aleatoria)]
  Discreta:
  \[
    X:S\rightarrow\{a_1,\dots,a_n\}\quad a_i\in\R\quad i=1,\dots,n
  \]
\end{definizione}

\paragraph{Funzione di probabilità di massa}%
\label{par:Funzione di probabilità di massa}

\paragraph{Funzione di probabilità cumulata}%
\label{par:Funzione di probabilità cumulata}

\section{Variabili aleatorie discrete}%
\label{sec:Variabili aleatorie discrete}
