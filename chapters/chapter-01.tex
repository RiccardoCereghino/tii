\chapter{Introduzione alla probabilità}%
\label{cha:Introduzione alla probabilità}
Lo studio della probabilità ha molteplici utilizzi:
\begin{itemize}
  \item analisi e design degli algoritmi;
  \item \textit{data science}, intelligenza artificiale.
\end{itemize}

\section{Calcolo combinatorio}%
\label{sec:Calcolo combinatorio}
Definiamo un \emph{esperimento} come un operazione che produce dei risultati.

Tipi di esperimenti sono:
\begin{itemize}
  \item permutazione;
  \item disposizione;
  \item combinazione.
\end{itemize}

Sia $N$ il numero di elementi di un insieme.

\paragraph{Permutazioni}%
\label{par:Permutazioni}
La permutazione rappresenta i possibili ordinamenti di $N$.
\[
  P(N)=N!
\]

\paragraph{Disposizioni}%
\label{par:Disposizioni}
La disposizione rappresenta i possibili ordinamenti di $i$ oggetti tra $N$.
\[
  D(i,N)= \frac{N!}{(N-i)!} 
\]

\paragraph{Combinazioni}%
\label{par:Combinazioni}
La combinazione rappresenta la scelta di $i$ oggetti da $N$.
\[
  C(i,N)= \frac{N!}{i!(N-i)!}
\]

\subimport{./chapter-01/}{section-01.tex}
